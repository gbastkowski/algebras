\documentclass{beamer}
\usepackage[utf8]{inputenc}
\usepackage[english]{babel}
\usepackage{cite}
\usepackage{tikz}
\usetikzlibrary{cd}
\usetheme{Berkeley}
\usecolortheme{spruce}

\title{Programming with Categories}
\subtitle{Yet another attempt to teach monads}
\author{Gunnar Bastkowski}
\date{\today}

\begin{document}
\maketitle
\frame{\tableofcontents[currentsection]}

\section{Introduction}


\section{Category Theory}
\begin{frame}
  \frametitle{What is Category Theory?}

  S. Awodey approaches the term {\it category theory} as follows\footnote{\cite{awodey2010category}}:
  \begin{quote}
    As a first approximation, one could say that category theory
    is the mathematical study of (abstract) {algebras of functions}.
  \end{quote}

  \begin{center}
    Category theory addresses the question:\\
    {\bf How can we combine functions to create new functions?}
  \end{center}
\end{frame}

\begin{frame}
  \frametitle{What is it good for?}
  \begin{itemize}
  \item Functional programming is about function composition
  \item Gives us tools to manipulate functions
  \item Provides a vocabulary like {\it Design Patterns}.\\
    But {\bf much} more powerful
  \end{itemize}

\end{frame}

\subsection{Vocabulary}
\begin{frame}[fragile]
  \frametitle{Vocabulary}
  \begin{itemize}
  \item objects ($A$, $B$) and Arrows ($f$) 
    \begin{flushright}
      \begin{tikzcd}
        A \arrow[r, "f"] & B \\
      \end{tikzcd}
    \end{flushright}
  \item composition ($g \circ f$)
    \begin{flushright}
      \begin{tikzcd}
        A \arrow[r, "f"] \arrow[rd, "g \circ f"'] & B \arrow[d, "g"] \\
                                              & C                \\
      \end{tikzcd}
    \end{flushright}
  \item identity ($id$)
    \begin{flushright}
      \begin{tikzcd}
        A \arrow[r, "id"] & A \\
      \end{tikzcd}
    \end{flushright}
  \end{itemize}
\end{frame}

\subsection{Laws}
\begin{frame}[fragile]
  \frametitle{Laws}
  \begin{itemize}
  \item arrows compose \\
    For every $f: A \to B, g: B \to C$, \\
    there exists a composition $g \circ f: A \to C$
  \item associativity \\
    $(A \to B) \to C = A \to (B \to C) = A \to B \to C$
  \item identity ($id$) \\
    For every $A_1, A_2$, with $A_1$ equal\footnote{up to isomorphism} to $A_2$,
    $f_1: A_1 \to A_2 = f_2: A_2 \to A_1 = id$
  \end{itemize}
\end{frame}

\subsection{Challenges}
\begin{frame}
  \frametitle{Challenges}
\end{frame}

\section{References}
\begin{frame}
  \frametitle{References}
  \begin{itemize}
  \item https://medium.com/@arjun.dhawan/composing-doobie-programs-5337695fd77b
  \item http://www.staff.city.ac.uk/~ross/papers/Applicative.pdf
  \item https://www.cs.ox.ac.uk/jeremy.gibbons/publications/iterator.pdf
  \end{itemize}
\end{frame}
\begin{frame}
  \bibliography{references}
  \bibliographystyle{plain}
\end{frame}
\end{document}